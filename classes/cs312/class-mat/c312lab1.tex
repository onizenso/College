% marc thomas
%
% macros:
%
\def\scriptA{{\cal A}}   % \cal for math mode only
\def\boldN{{\bf N}}
\def\boldR{{\bf R}}
\def\boldRP{{\bf R}$^+$}   % bold-face R plus
\def\boldC{{\bf C}}
%
\def\eop{\vrule height10pt width6pt depth0pt}
\def\mapleft#1{\ \smash{\mathop{\longleftarrow}\limits^{#1}}\ }
%
% end of macros:
%
% font declarations for larger fonts:
%
\font\bigcal=cmsy10 at 17.28pt
\font\bigrm=cmr10 at 17.28pt
%
\magnification=\magstep1   % magnify output chosen number of steps
\nopagenumbers             % no pagenumbers set
\voffset -0.5 true in
\hoffset 0.0 true in
\hsize = 6.5 true in       % true page size
\vsize = 9.8 true in       % if using pagenumbers reduce this
\baselineskip12pt          % line spacing
\parskip4pt                % space between paragraphs
\null
\centerline{\bf COMPSCI 312 -- LAB 1}
\line{\hfill Marc Thomas}
\vskip5pt

\noindent
   This lab concerns creating random data and
coding algorthms in {\tt C} and {\tt C++}.

\noindent
1. Make a subdirectory named ``{\tt cs312}'' under your HOME directory 
  and make yourself a copy of the (ascii text) class
  file: {\tt c\_and\_c++.txt} from the class directory: \hfil\break
\centerline{\tt /home/fac/marc/public\_html/code/cs312} 
Read this informational file. Why do you think that it often
said that ``{\tt C++} is a superset of {\tt C}?'' Additional
information is given in the class file: {\tt C\_diffs.txt}.

\noindent
2. Make yourself a copy of the class file (and {\tt C++} program): 
 {\tt uniform.cpp} and
 you may also find it useful to also make a copy of: {\tt Makefile}. 
 Compile and link the program either by typing: \hfil\break
 {\tt g++ -g uniform.cpp -o uniform} \hfill\break
 or (more easily) by using the {\tt Makefile}: \hfil\break
 {\tt make uniform} \hfil\break
 {\tt make clean} \hfil\break
 The program {\tt uniform} allows you to generate a set of $ n $ random 
 numbers (written to {\tt stderr}) chosen from 
 some interval $ [a, b] $. You can 
 save the numbers in the usual way by redirecting them to a file. Since
 {\tt stderr} is file descriptor 2, this would be done, for example, by
 typing: \hfil\break
 {\tt uniform 2$>$ my\_datafile.txt} \hfil\break
 Note that the program always writes the {\tt sentinel} {\tt 0.0}
 at the end of the data. Why? You can, of course, remove this
 or modify the program. Make a datafile of about 30--40 random
 numbers chosen from the interval $[-5.0 , 5.0]$ for use later.

\noindent
3. Make yourself a copy of the class file (and {\tt C++} program): 
 {\tt sorting.cpp}.
 This program will accept a datafile of floating point numbers
 (separated by white space) and do some statistical operations
 on the data set.  A sentinel is {\bf not} necessary with this program. 
 Compile and link the program either by typing: \hfil\break
 {\tt g++ -g sorting.cpp -o sorting} \hfill\break
  or (more easily) by using the {\tt Makefile}: \hfil\break
 {\tt make sorting} \hfil\break
 {\tt make clean} \hfil\break
 Run the program on your datafile via: \hfil\break
 {\tt sorting my\_datafile.txt} \hfil\break
 Try out all operations provided in the menu and make sure that
 the answers are correct. We will eventually
 discuss all of the sorting algorithms (selection, insertion,
 heap, merge, quick, and shell sort) during the course of the
 quarter.

\vskip10pt
\noindent
{\bf Questions} Write up a short explanation of your conclusions
 and observations (no more than one page) on items 1--3 above.
 
\noindent
Email me your explanation {\bf in plain ascii text} observing
 the following:
\itemitem{i.} keep line length under 80 characters per line and
	{\bf don't send quoted-printable text} (which is painful to
	read since each line ends in an `=' sign and it is full
	of stupid =0A's and =20's).
\itemitem{ii.} send the text {\bf in-line} and {\bf not as an
	attachment} (e.g. in pine you can use CTRL-R to read text
	into the email body).
\itemitem{iii.} if you need to refer to files that you have in your
	account, just give me the path to the file and the filename.
	{\bf Don't} use spaces in filenames; you can use the underscore
	character to improve readability.

\vfill
\eject
\end
