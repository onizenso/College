% marc thomas
%
% macros:
%
\def\scriptA{{\cal A}}   % \cal for math mode only
\def\boldN{{\bf N}}
\def\boldR{{\bf R}}
\def\boldRP{{\bf R}$^+$}   % bold-face R plus
\def\boldC{{\bf C}}
%
\def\eop{\vrule height10pt width6pt depth0pt}
\def\mapleft#1{\ \smash{\mathop{\longleftarrow}\limits^{#1}}\ }
%
% end of macros:
%
% font declarations for larger fonts:
%
\font\bigcal=cmsy10 at 17.28pt
\font\bigrm=cmr10 at 17.28pt
%
\magnification=\magstep1   % magnify output chosen number of steps
\nopagenumbers             % no pagenumbers set
\voffset -0.5 true in
\hoffset 0.0 true in
\hsize = 6.5 true in       % true page size
\vsize = 9.8 true in       % if using pagenumbers reduce this
\baselineskip12pt          % line spacing
\parskip4pt                % space between paragraphs
\null
\centerline{\bf COMPSCI 312 -- LAB 2}
\line{\hfill Marc Thomas}
\vskip5pt

\noindent
   This lab concerns timing {\tt C} and {\tt C++} programs
and getting values for both {\tt user} and {\tt system} times.
Note that {\tt user} time is the same as {\tt virtual} time and
that {\tt profiling} time is the sum of the {\tt user} and
{\tt system} times. The  {\tt real} time is not very useful because
it depends too much on machine load.

\noindent
1. {\bf First method:} Run your program using the ``wrapper:'' \ \ 
 {\tt /usr/bin/time -p}. \hfil\break
 This will run your program using the system calls {\tt fork()}
 and {\tt execve()} and then print out the {\tt real}, 
 {\tt user}, and {\tt system} times after the program ends. 
 Note that the {\tt -p} option is recommended so that you
 get the timing information in Posix format. If your program
 requires user input you can use a Unix pipe. For example, if
 you want to do a quicksort on the large dataset {\tt datafile6.txt}
 you could type: \hfil\break
{\tt sorting datafile6.txt} \hfil\break
 and manually enter the keystrokes ``{\tt q}'' for quicksort 
 and ``{\tt x}'' to quit. 
 But, you could time this as one atomic operation as follows: \hfil\break
{\tt echo "q x" | /usr/bin/time -p sorting datafile6.txt} \hfil\break
 The format of the full command line in general will be: \hfil\break
{\tt echo "$\{$keystrokes$\}$" | /usr/bin/time -p $\{$command with
   any parameters$\}$ }\hfil\break
 {\bf Note that:}
\itemitem{i.} times reported by the operating system are usually
	only accurate to $\pm$5--10\%, depending on the
	system, so you may have to do several runs and take an 
	average if you need better accuracy.
\itemitem{ii.} if the program executes with a time under a hundredth 
	of a second you will get $0.00$ as an answer, so this method is
	not useful for quickly executing programs.

\noindent
 Time all of the different sorting options available in the
 {\tt sorting} program on the large dataset {\tt datafile6.txt}.
 Are the large variations in time you get surprising?

\noindent
2. {\bf Second method:} Set timers in your programs, using the timing
 system calls \hfil\break
 {\tt setitimer()}, {\tt getitimer()}, etc., or
 the newer POSIX calls {\tt timer\_create()}, \hfil\break
 {\tt timer\_settime()}, etc.
 To say that these calls are complicated is an {\it understatement} 
 but I have written
 the {\tt C} module {\tt timing.c} with header file {\tt timing.h} which
 you can use and which is much simpler. 
 To see an example of this, get the sample program {\tt cache.c}.
 Look at the source code, particularly the lines: \hfill\break
{\tt \#include "timing.h"} \hfil\break
{\tt $\ldots$ } \hfil\break
{\tt init\_timing(); /* initialize and start timers */} \hfil\break
{\tt $\ldots$ } \hfil\break
{\tt read\_timing(); /* get the results */} \hfil\break
{\tt pause\_timing(); /* stop the timers */} \hfil\break
{\tt $\ldots$ } \hfil\break
{\tt fprintf\_timing(); /* write the results to stdout or file */} \hfil\break
{\tt $\ldots$ } \hfil\break
 Compile the program (with the {\tt timing} module) either via: \hfil\break
{\tt gcc -g cache.c timing.c -o cache} \hfil\break
 or by using the makefile: \hfil\break
{\tt make cache} \hfil\break
{\tt make clean} \hfil\break
 This program tests the effective speed of the L2~cache by reading and
 writing a large amount of information to an array using a particular
 {\tt line\_offset}. It then uses the timing results to compute the
 net memory bandwith in megabytes per second for that
 {\tt line\_offset}. Run it, trying some numbers from 127--4096 and
 see if you can discern a pattern.

\vskip10pt
\noindent
{\bf Assignment} Write a {\tt C} or {\tt C++} program which  will
 accept a positive integer {\tt n} from 2--2000000 on the command line and 
 which will display the first 20 primes greater than or equal to {\tt n}.
 For an example executable, copy over {\tt cpubound} from the class
 directory and run it, for example, with {\tt n=2000}: \hfil\break
{\tt cpubound 2000} \hfil\break
 Time you program for values of {\tt n=200, 2000, 20000} and {\tt 200000}
 using Method~1 (with {\tt /usr/bin/time -p}) above.
 
\noindent
Email me your timings ({\tt user} and {\tt system}) for the
values of {\tt n} above and the
{\bf path} (and {\bf only the path}) to your solution. For example, if your
program is named {\tt lab2.c} you might email me that \hfil\break
{\tt my solution is at: /usr/stu/myusername/cs312/lab2.c } \hfil\break
Please:
\itemitem{i.} Keep line length under 80 characters per line.
\itemitem{ii.} Do not send any attachments.

\vfill
\eject
\end
