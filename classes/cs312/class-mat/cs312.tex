% marc thomas
%
% define the following macros:
%
\def\scriptA{{\cal A}}   % \cal for math mode only and no 
\def\boldN{{\bf N}}
\def\boldR{{\bf R}}
\def\boldRP{{\bf R}$^+$}   % bold-face R plus
\def\boldC{{\bf C}}
%
% end of macros
%

\magnification=\magstep1   % magnify output one step
\nopagenumbers
\voffset -0.5 true in
\hsize = 6.5 true in       % but keep same page size
\vsize = 9.8 true in
\baselineskip12pt
\parskip2pt

\noindent 
{\bf Course Description} \hfill\break
{\bf CMPS 312} 

\centerline{\bf Algorithm Analysis and Design}
\vskip5pt

\noindent
{\bf Catalog Description:}
	\itemitem{} Algorithm analysis, asymptotic notation, hashing, 
	hash tables, scatter tables, and AVL and B-trees, brute-force 
	and greedy algorithms, divide-and-conquer algorithms, dynamic 
	programming, randomized algorithms, graphs and graph algorithms, 
	and distributed algorithms.

\noindent
{\bf Prerequisite:}
	\itemitem{} CMPS 223 and CMPS 295/300

\noindent
{\bf Units:}
	\itemitem{} 5

\noindent
{\bf Instructor:} 
	\itemitem{} Marc Thomas

\noindent
{\bf ACM/IEEE Body of Knowledge Topics:}
	\itemitem{} (CS/CE-AL1) Basic algorithmic analysis.
	\itemitem{} (CS/CE-AL2) Algorithmic strategies.
	\itemitem{} (CS/CE-AL3) Fundamental computing algorithms.
	\itemitem{} (CS/CE-AL4) An introduction to distributed algorithms.
	\itemitem{} (CE-AL5) Algorithmic complexity.
	\itemitem{} (Laboratory) Become proficient in writing
	  programs implementing algorithms, understanding memory usage,
	  user and system time, and impact of caches and virtual memory
	  on performance.

\noindent
{\bf Texts:}
	\itemitem{} Required: 
	Anany Levitin {\it Introduction to the Design and Analysis
	of Algorithms} (3rd edition, {\tt ISBN-10: 0-13-231681-1})

\noindent
{\bf Topics (roughly) by Week:}
	\itemitem{}  (1.1--1.4 and Appendix~A) introduction, basic
	  issues with examples, useful formulas, important problem types, 
	  and review of data structures.
	\itemitem{}  (2.1--2.3) input size, order of growth and big-{\cal O}
	  notation, analysis and examples of non-recursive algorithms,
	\itemitem{}  (2.4, Appendix~B, and 3.1--3.2) analysis and
	  examples of recursive algorithms,
	  searching, sorting, and string matching.
	\itemitem{}  (3.3--3.4) closest-pair, convex hull problems,
	  and exhaustive search (traveling salesman, knapsack).
	\itemitem{}  (4.1--4.3) insertion sort, topological sorting,
	  and combinatorial considerations.
	\itemitem{}  (4.4) binary search (5.1--5.2) mergesort and quicksort.
	\itemitem{}  (5.3--5.4) binary tree traversals.
	\itemitem{}  (6.2--6.4) Gaussian elimination, AVL and 2--3 
	  search trees, and heapsort.
	\itemitem{}  (6.5) Horner's rule (7.3--7.4) hashing and B-trees.
	\itemitem{}  Additional topics (e.g. multi-threaded code, and
	  distributed algorithms).

\noindent
{\bf Laboratory:}
	\itemitem{} The laboratory session will parallel the 
	lecture, illustrating the principles and familiarizing the
	student with actual coding and hardware performance. Coding
	will be in {\bf both} the {\tt C} and {\tt C++} languages and 
	we will cover the use of timing 
	({\tt user}, {\tt system}, {\tt prof}) routines.

\noindent
{\bf ABET Outcome Coverage:}
  \itemitem{3a.} {\it An ability to apply knowledge of computing and mathematics
        appropriate to the discipline. An ability to understand how 
        computer science relates to mathematics and the physical sciences.}
	  Laboratory/homework assignments and questions on the 
	midterms and final require direct applications of the 
	mathematical theory of algorithms pertinent to computer science.
  \itemitem{3b.} {\it An ability to analyze a problem, and identify and define
        the computing requirements and specifications appropriate to 
	its solution.} 
	  Implementation on actual hardware and the ability to analyze
	performance (e.g. the roles of caches, virtual memory, use of
	multi-threaded code) will be required for successful completion
	of laboratory/homework assignements and will be tested on 
	the exams.
  \itemitem{3j.} {\it An ability to apply mathematical foundations, algorithmic
        principles, and computer science theory in the modeling and 
        design of computer-based systems in a way that demonstrates
        comprehension of the tradeoffs involved in design choices.}
	  Implementation and performance analysis of different algorithms 
	(e.g. direct, recursive, etc.) which solve the same problem will 
	be required for successful completion of laboratory/homework
	assignments and will be tested on the exams.
  
\noindent 
{\bf Grading:} 
	\itemitem{} Two midterms will be given, each worth 25\%.
	I do not give make-up midterms; for an excused absence I count the
	other grades proportionately higher.  One final exam, comprehensive
	but emphasizing the later material, will be given. It is mandatory
	and worth 25\%. Homework and lab work are together worth the
	remaining 25\%. 

\noindent
{\bf Attendance Policy:}
	\itemitem{} Students are responsible for their own attendance. 
	Course materials and assignments will be posted on the course
	website: \hfil\break
	{\tt http://www.cs.csubak.edu/\~{}marc/code/cs312.html}

\noindent
{\bf Academic Integrity Policy:}
	\itemitem{} Homeworks and labs may be worked on and strategy
	discussed in groups. However, unless otherwise stated, all 
	assignments are {\it individual} assignments in that each 
	student must turn in his/her own work; {\bf no} direct copying 
	is allowed.  Refer to the Academic Integrity policy printed in 
	the campus catalog and class schedule.

\noindent
{\bf Tutoring Center:}
	\itemitem{} The Tutoring Center in Sci~III 324 is available for 
	use by students in this course outside of class time on a first 
	come, first serve basis.  Priority in the lab is given to students 
	who are completing assignments for department courses. The hours 
	for the Tutoring Center are posted on the department website.

	\itemitem{} Students in this course may ask the tutors for 
	assistance on assignments.  The tutors are not allowed to solve 
	the assignment for you, but they can assist with problems like 
	cryptic compiler errors.

\vfill
\eject
\end
